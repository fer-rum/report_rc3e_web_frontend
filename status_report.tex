\documentclass{beamer}

\usepackage[utf8]{inputenc}
\usepackage{hyperref}
\usepackage{tikz}
\usetikzlibrary{positioning}
\usetikzlibrary{calc}
\usetikzlibrary{arrows.meta}
\usetikzlibrary{trees}


\usepackage{tikzscale}
\usepackage{graphicx}
\usepackage{listings}

\usetheme[pagenum, beamerfont]{tud}

\AtBeginSection{\frame{\sectionpage}}

\newcommand{\rccce}{\texttt{RC3E}}
\newcommand{\django}{\texttt{Django}}

\title{Design and Implementaion of a Web-Based \rccce{} User Frontend}
\subtitle{Status Report}
\author{Fredo Erxleben}
\date{\today}
\institute{Chair for VLSI Design, Diagnostics and Architecture}

\lstdefinestyle{http}{
  emptylines=1,
  breaklines=true,
  basicstyle=\tiny\ttfamily\color{black},
  moredelim=**[is][\color{blue}]{@}{@},
  moredelim=**[is][\color{gray}]{@!}{!@},
}

\begin{document}

\maketitle

    \begin{frame}{Contents of this Talk}
        \tableofcontents
    \end{frame}


\section{Introduction}
\subsection{\rccce{}}

    \begin{frame}{The Core Issue}
        \rccce{} (
            \textbf{R}econfigurable 
            \textbf{C}ommon 
            \textbf{C}loud 
            \textbf{C}omputing 
            \textbf{E}nvironment
        ) \\
        \begin{itemize}
            \item System for distribution of FPGA ressources to remote users
            \item Actual ressources get abstracted as virtual FPGAs
            \item Controlled by a human administrator, who\dots
            \begin{itemize}
                \item \dots runs selected scripts on the host
                \item \dots manages users by hand
                \item \dots has to schedule ressources
                \item \dots does the programming of vFPGAs and relays the results
            \end{itemize}
        \end{itemize}

        \begin{block}{Project Aim}
            Create a web-frontend that can automatize these tasks.
        \end{block}
    \end{frame}

\subsection{\django{}}

    \begin{frame}{What is \django?}
        \begin{itemize}
            \item \texttt{Python}-based framework for web applications
            \item Architecture is \textbf{M}odel \textbf{T}emplate \textbf{V}iew
	      \begin{itemize}
	       \item  Variation of MVC pattern 
	      \end{itemize}
            \item Focus on reuseability, rapid prototyping and the \emph{DRY}-principle\footnote{\emph{DRY} = Don't repeat yourself}
            \item Huge amount of out-of-the-box features
        \end{itemize}
    \end{frame}


    %\begin{frame}{Selected features}
    %    \begin{itemize}
    %        \item Integrated web-server for development and testing
    %        \item Automatic HTML form serialization and validation
    %        \item HTML generation from parametrized templates (+Template inheritance)
    %        \item Build-in authentification and admin interface
    %    \end{itemize}
    %\end{frame}

    \begin{frame}[fragile]{Architecture of a \django{} Project}
        \begin{figure}
            \centering
            \includegraphics[]{django_project_structure.tikz}
        \end{figure}
    \end{frame}

    \begin{frame}[fragile]{Operation Principle of a \django{}-App}
        \begin{figure}
            \centering
            \includegraphics[]{django_architecture.tikz}
        \end{figure}
    \end{frame}

\subsection{REST}

    \begin{frame}{What is REST?}
        \textbf{RE}presentational \textbf{S}tate \textbf{T}ransfer\\
        \begin{itemize}
            \item Architecture principle for distributed web applications 
            \item Allow clients to access a textual representation of a ressource on the host
            \item Set of available operations on the ressources are well-defined by a simple interface
        \end{itemize}
        
        \begin{exampleblock}{Benefit}
            Easy to parse for machine-to-machine communication. \\
            Useful for interaction with the \rccce -server.
        \end{exampleblock}

    \end{frame}
        
    \begin{frame}[fragile]{Example for a RESTful API Call}
        Request:
        \begin{lstlisting}[style=http]
        fredo in ~ >  http @GET localhost:8000/rest_api/producers/@
        \end{lstlisting}
        
        Response:
        \begin{lstlisting}[style=http]
        HTTP/1.1 200 OK
        Content-Type: @application/json@
        @![More http header stuff]!@
        @[
            {
                "id": 1, 
                "name": "Xilinx"
            }, 
            {
                "id": 2, 
                "name": "Altera"
            }, 
            {
                "id": 3, 
                "name": "Intel"
            }
        ]@
        \end{lstlisting}
    \end{frame}

\section{Data Model}

    \begin{frame}{Preface}
        The data model is the backbone of the project.
        \begin{itemize}
            \item Must be represented by a relational DB
            \item Fine granularity to improve changeability
        \end{itemize}
        
        \begin{alertblock}{Important!}
            FPGAs are assumed to be homogenous.
        \end{alertblock}

        \begin{block}{Note}
            Multiple development iterations introduced some technical debt.\\
            But: It can be purged by refactoring cycles.
        \end{block}

    \end{frame}

    \begin{frame}[fragile, allowframebreaks]{Data Model Details}
        \begin{figure}
            \centering
            \includegraphics[]{data_model_core.tikz}
        \end{figure}
        
    \framebreak
        \begin{figure}
            \centering
            \includegraphics[]{data_model_reservation.tikz}
        \end{figure}    

    \framebreak
        \begin{figure}
            \centering
            \includegraphics[]{data_model_scripting.tikz}
        \end{figure}
        
    \end{frame}

\section{Selected Issues}

\subsection{User Interaction Concept}

    \begin{frame}[allowframebreaks]{Basic Principle of User Interaction}
        Boil down all interaction to the following operations:
        \begin{itemize}
            \item \emph{List} objects of a type
            \item \emph{Show} object details
            \item \emph{Create} objects
            \item \emph{Delete} objects
        \end{itemize}

        \begin{block}{Note}
            Allowing to modify existing objects has severe implications! \\
            (What happens when you change the region type/ count of an FPGA?)
        \end{block}
        
    \framebreak
    
        \begin{alertblock}{Issue}
            URLs and internal project structure have to consistently represent this principle.
        \end{alertblock}
    
        \begin{alertblock}{Issue}
            Look and feel of the HTML pages should be uniform.
        \end{alertblock}
    
        \begin{exampleblock}{Solution}
            Use \django s HTML template inheritance.
        \end{exampleblock}

        \begin{block}{Note}
            Even special pages can profit from inheriting layout templates.
        \end{block}
    
    \end{frame}
    
    \begin{frame}[fragile]{HTML Template Structure}
        \begin{figure}
            \centering
            \includegraphics[]{html_template_structure.tikz}
        \end{figure}
    \end{frame}
    
    \begin{frame}{Automating URL-Binding Generation}
        \begin{alertblock}{Issue}
            Generating all the URL-bindings for the \emph{List/Show/Create/Delete} pages by hand is tedious and error-prone.
        \end{alertblock}

        \begin{exampleblock}{Solution}
            Write generator functions for the common case and only deal with the special cases by hand.
        \end{exampleblock}
        
        \begin{block}{Note}
            Generator functions also check if the required templates are present.
        \end{block}

        \tiny{(PS: \url{https://www.xkcd.com/917/})}
    \end{frame}

\subsection{Supporting Multiple APIs}
    
    \begin{frame}[fragile, allowframebreaks]{Supporting Multiple APIs}
        \begin{alertblock}{Issue}
            Need to support a REST and a web API
        \end{alertblock}

        \begin{itemize}
            \item Both use the same protocol (HTTP/ HTTPS)
            \item HTTP request types \emph{GET}, \emph{POST} used by web API
            \item REST will use these as well
        \end{itemize}
    
        How to figure out what to do with an HTTP request?
     
        \framebreak
    
        \begin{block}{Idea} 
            Give each API a different URL namespace.
        \end{block}
    
        Web-API:
        \begin{lstlisting}[style=http]
            @GET    /web_api/@producers/create/
            @POST   /web_api/@producers/create/
        \end{lstlisting}
    
        REST-API:
        \begin{lstlisting}[style=http]
            @GET    /rest_api/@producers/
            @POST   /rest_api/@producers/
            @GET    /rest_api/@producers/1/
            @DELETE /rest_api/@producers/1/
        \end{lstlisting}
    
        \framebreak
    
        \begin{alertblock}{Issue}
            The REST-framework alters the \django{}-apps structure.
        \end{alertblock}
        
        \begin{exampleblock}{Solution}
            Split the project into one app per API.
        \end{exampleblock}

        \begin{alertblock}{Next Issue}
            Both apps have to operate on the exact same model.
        \end{alertblock}
        
        \begin{exampleblock}{Solution}
            Split off the model into a separate app as well.
        \end{exampleblock}
    
    \framebreak
    
        \begin{figure}
            \centering
            \includegraphics[]{modified_project_structure.tikz}
        \end{figure}
    
    \end{frame}

\subsection{Reservation Scheduling}
    
    \begin{frame}[fragile, allowframebreaks]{Scheduling of Reservations}
        \begin{block}{User Input}
            \begin{itemize}
                \item Region type
                \item Region count (consecutive)
                \item Desired start and end of reservation
            \end{itemize}
        \end{block}

    \framebreak
        
        \emph{View} generates and issues \texttt{SQL} query from user input.
        
        \begin{block}{Database Action}
            \begin{enumerate}
                \item Filter for FPGAs with correct region type
                \item Filter regions without reservation in desired timeframe
                \item Sort remaining regions by index 
            \end{enumerate}
        \end{block}
        
        Result type: \texttt{\{FPGA: [Region]\}}

    \framebreak
        \emph{View} then processes the query result.
    
        \begin{block}{Pseudocode}
         
        \begin{lstlisting}[style=http]
            @foreach@ fpga:
            @!|!@   @foreach@ region:
            @!|!@   @!|!@   @if@ region is not consecutive:
            @!|!@   @!|!@   @!|!@   reset markers @!// deal with fragmentation!@
            @!|!@   @!|!@
            @!|!@   @!|!@   mark region
            @!|!@   @!|!@   
            @!|!@   @!|!@   @if@ enough regions marked:
            @!|!@   @!|!@   @!|!@   break all @!// Got enough reservation candidates!@
            @!|!@
            @!|!@   reset markers @!// Reservations across FPGAs are not allowed!@

            reserve candidates @!// If there are any!@
            @!// Inform User!@
        \end{lstlisting}

        \end{block}

     \framebreak
     
     \begin{block}{Notes}
         \begin{itemize}
             \item \emph{First fit} scheduling does not use ressources equally
             \item Fragmentation may become an issue in real-life use
             \item Reservations are not supposed to be deleted (accounting) which may slow down the DB after many reservation cycles 
         \end{itemize}
     \end{block}
     
     It works for the scale of \rccce{} but has optimization potential.
        
    \end{frame}


\section{Results and Future Work}

    \begin{frame}{Summary of Current State}
    
        \begin{itemize}
            \item Data model has been developed
            \item Technology research is completed
            \item Model is implemented
            \item Web-API is implemented
            \item Currently known issues have been solved
        \end{itemize}
    
    \end{frame}

    \begin{frame}{Future Work}
    
        Immediate tasks:
        \begin{itemize}
            \item Finish transition to new HTML template structure
            \item Finish implementation of REST-API
        \end{itemize}
        
        Whishlist:
        \begin{itemize}
            \item Subsystem for uploading designs, running the FPGA programming scripts and operation feedback
            \item Subsystem for statistics and accounting
            \item Logging \rccce{} interactions to DB
        \end{itemize}

        \begin{block}{Ultimative Goal}
            Connect the frontend with the live \rccce -system.
        \end{block}
     
    \end{frame}

\appendix
\section{Appendix}

    \begin{frame}[allowframebreaks]{Further Reading}
    
        \begin{block}{\rccce}
            \emph{RC3E: Reconfigurable Accelerators in Data Centres and Their Provision by Adapted Service Models},
            Oliver Knodel, Patrick Lehmann, Rainer G. Spallek, 2016\\
            \url{https://ieeexplore.ieee.org/abstract/document/7820030/}
            \vspace{1 em}
        
            \emph{Rekonfigurierbare Hardwarekomponenten im Kontext von Cloud-Architekturen}, 
            Dissertation of Oliver Knodel, 2018 \\
            \color{gray}{Publication pending}
        \end{block}
    
    \framebreak
    
        \begin{block}{REST}
            \emph{Architectural Styles and the Design of Network-based Software Architectures},
            Dissertation of Roy Fielding, 2000\\
            \url{https://www.ics.uci.edu/~fielding/pubs/dissertation/fielding_dissertation.pdf}
        \end{block}
        
    \framebreak
    
        \begin{block}{\django}
            \emph{Official Documentation} \\
            \url{https://docs.djangoproject.com/en/2.0/} \\
            \vspace{1em}
            \emph{Django-REST} \\
            \url{http://www.django-rest-framework.org/}
        \end{block}

    \end{frame}
    
    \begin{frame}{Last Slide}
        
        Thank you for your attention. \\
        Questions?\\
        
        \vspace{1em}
        
        \begin{tabular}{ll}
            \textbf{Office:}    & TUD, APB 1095\\
            \textbf{Phone:}     & TUD-38456 \\
            \textbf{E-Mail:}    & \href{mailto:fredo.erxleben@tu-dresden.de}{fredo.erxleben@tu-dresden.de} \\
            \textbf{Project Git:}   & \url{http://github.com/VLSI-EDA/rc3e-manager}
        \end{tabular}

        \vspace{1em}
        
        \footnotesize{PS: We are looking for students to continue working on this project.}\\
        \vspace{0.5em}
        \tiny{Bachelor thesis, anyone? ;)}
        
    \end{frame}

\end{document}
